\subsection{R/\texttt{iprobit}}

\begin{frame}{Variational I-prior probit}
  \begin{tikzpicture}[scale=1.1, transform shape]
    \tikzstyle{main}=[circle, minimum size = 10mm, thick, draw =black!80, node distance = 16mm]
    \tikzstyle{connect}=[-latex, thick]
    \tikzstyle{box}=[rectangle, draw=black!100]
      \node[main, draw=none] (fake) {};
      \node[main, fill = black!10] (H) [right=of fake, xshift=-1.65cm] {$x_i$};
      \node[main, double, double distance=0.6mm] (eta) [right=of H] {$f_i$};
      \node[main, draw=colpur] (ystar) [right=of eta] {\color{colpur} $y_i^*$};
      \node[main, draw=colblu] (lambda) [above=of H, xshift=0.4cm, yshift=-0.4cm] {\color{colblu} $\lambda$};  
      \node[main, draw=colgre] (alpha) [above=of eta, yshift=0.3cm] {\color{colgre} $\alpha$};  
      \node[main, fill = black!10] (y) [right=of ystar] {$y_i$};
      \node[main, draw=colred] (w) [below=of eta, yshift=0.4cm] {\color{colred} $w_i$};  
      \path (alpha) edge [connect] (eta)
            (lambda) edge [connect] (eta)
    		(H) edge [connect] (eta) 
    		(eta) edge [connect] (ystar)
    		(ystar) edge [connect] (y)
    		(w) edge [connect] (eta)
            (H) edge [] node [above] {$h$} (eta);
      \node[rectangle, draw=black!100, fit= (H) (y) (w) ] {}; 
      \node[rectangle, fit= (w) (y), label=below right:{$i=1,\dots,n$}, xshift=-0.35cm, yshift=0.55cm] {};  % the label
    \end{tikzpicture}
    
    \begin{textblock*}{0.48\textwidth}(0.52\textwidth,0.55cm)
    \begin{block}{}
    \vspace{-1.6em}
      \begin{align*}
        &p(\by,\by^*,\bw,\alpha,\lambda)  \\
        &= p(\by|\by^*)p(\by^*|\bff)p(\bw)p(\lambda)p(\alpha) \\
        &= {\textstyle\prod_{i=1}^n} \ind[y_i^* \geq 0]^{y_i} \ind[y_i^* < 0]^{1-y_i} \\
        &\phantom{==} \cdot {\color{colpur} {\textstyle\prod_{i=1}^n} \{ \N(f_i,1) \}} \cdot {\color{colred} [\N(0,1)]^n} \\
        &\phantom{==} \cdot {\color{colblu} \N(\lambda_0,\kappa_0^{-1})} \cdot  {\color{colgre} \N(\alpha_0,\tau_0^{-1})}
      \end{align*}
    \end{block}
  \end{textblock*}
\end{frame}

\begin{frame}{Posterior distribution}
  \begin{itemize}
    \item Approximate the posterior by a mean-field variational density
    \[
      p(\by^*,\bw,\alpha,\lambda|\by) \approx \prod_{i=1}^n q(y_i^*)q(\bw)q(\alpha)q(\lambda)\vspace{-2pt}
    \]
    where
  \end{itemize}
  \vspace{-15pt}
  \begin{center}   
    \begin{gather*}
      q(y_i^*) \equiv 
      \begin{cases}
        \ind[y_i^* \geq 0] \N(\tilde f_i, 1) &\text{ if } y_i = 1\\
        \ind[y_i^* < 0] \N(\tilde f_i, 1) &\text{ if } y_i = 0\\
      \end{cases}
    \hspace{1cm}
    q(\bw) \equiv \N(\tilde\bw, \tilde\bV_w) \\[0.5em]
    q(\lambda) \equiv \N(\tilde\lambda, \tilde v_w)
    \hspace{1cm}  
    q(\alpha) \equiv \N(\tilde\alpha,1/n) \\[0.5em]
    \tilde f_i = \tilde\alpha + {\textstyle\sum_{k=1}^n} h_{\tilde\lambda}(x_i, x_k)\tilde w_k 
    \hspace{1cm} 
    \tilde\alpha = \frac{1}{n}{\textstyle\sum_{k=1}^n} \left( \E[\by^*_i] - h_{\tilde\lambda}(x_i, x_k)\tilde w_k \right) \\[0.3em]
    \tilde \bw = \tilde\bV_w\bH_{\tilde\lambda}(\E[\by^*] - \tilde\alpha\bone_n)
    \hspace{1cm} 
    \tilde\bV_w^{-1} = \bH_{\tilde\lambda}^2 + \bI_n \\[0.3em]
    \tilde\lambda = (\E[\by^*] - \tilde\alpha\bone_n)\bH\tilde\bw / \tilde v_\lambda
    \hspace{1cm} 
    \tilde v_\lambda = \tr(\bH^2(\tilde\bV_w + \tilde\bw\tilde\bw^\top))
    \end{gather*}
  \end{center} 
\end{frame}

\begin{frame}{Posterior predictive distribution}
  \begin{itemize}\setlength\itemsep{1em}
    \item Given new data points $x_{\text{new}}$, interested in
    \begin{align*}
      p(y_{\text{new}}|\by) &= \int p(y_{\text{new}} | y^*_{\text{new}}, \by) p (y^*_{\text{new}} | \by) \d y^*_{\text{new}} \\
      &\approx \int p(y_{\text{new}} | y^*_{\text{new}}) q (y^*_{\text{new}}) \d y^*_{\text{new}} \\
      &= \begin{cases}
        \Phi(\tilde f_{\text{new}}) & \text{ if } y_{\text{new}} = 1 \\
        1 - \Phi(\tilde f_{\text{new}}) & \text{ if } y_{\text{new}} = 0 \\
      \end{cases}
    \end{align*}
    where $\tilde f_{\text{new}} = \tilde\alpha + {\sum_{k=1}^n} h_{\tilde\lambda}(x_{\text{new}}, x_k)\tilde w_k$.
    \item $f_{\text{new}}$ represents the estimate of the latent propensity for $y_{\text{new}}$, and its uncertainty is described by $q(y_{\text{new}}^*)$.
  \end{itemize}
\end{frame}

\begin{frame}{Variational lower bound}
  \begin{itemize}\setlength\itemsep{1em}
    \item Since the solutions are coupled, we implement an iterative scheme \\ (as per Algorithm \ref{alg:cavi})
    \item Assess convergence by monitoring the lower bound
    \begin{align*}
      \cL 
      &= \E_q[\log p(\by,\by^*,\bw,\alpha,\lambda)] - \E_q[\log q(\by^*,\bw,\alpha,\lambda)] \\
      &= \const + \sum_{i=1}^n \left(y_i \log \Phi(\tilde f_i) + (1-y_i) \log \big(1 - \Phi(\tilde f_i)\big)\right) \\
      &\phantom{==}-\half \left( \tr\tilde\bV_w + \tr (\tilde\bw\tilde\bw^\top) - \log \vert \tilde\bV_w \vert + \log \tilde v_\lambda \right) 
    \end{align*}
    \item ISSUE: Different initialisation leads to different converged lower bound values indicating presence of many local optima.
  \end{itemize}
\end{frame}

\begin{frame}{R/\texttt{iprobit}}
  \blfootnote{\fullcite{Jamil2017iprobit}}
\end{frame}

\subsection{Examples}

\begin{frame}{Fisher's Iris data set}
1. Intro. Combine some groups so binary classification problem. For illustration just use sepal length and width (to get nice plots).
2. Fit model. Syntax. Summary.
3. Multiple starting values leads to different L.
4. Plot LB. Plot decision boundary.
\end{frame}

\begin{frame}{Cardiac arrhythmia data set}
1. Intro. Number of covariates.
2. Subsample, fit and get SE for out-of-sample test error rates.
3. Compare with other classifiers.
\end{frame}

\begin{frame}{Multilevel example}
Not sure what yet. Something that latent propensities might be worth measuring? Maybe fitted probabilities too.
\end{frame}
















