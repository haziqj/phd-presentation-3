\subsection{The I-prior probit model}
\begin{frame}{The I-prior probit model}
  \begin{tikzpicture}[scale=1.1, transform shape]
    \tikzstyle{main}=[circle, minimum size = 10mm, thick, draw =black!80, node distance = 16mm]
    \tikzstyle{connect}=[-latex, thick]
    \tikzstyle{box}=[rectangle, draw=black!100]
      \node[main, draw=none] (fake) {};
      \node[main, fill = black!10] (H) [right=of fake, xshift=-1.65cm] {$x_i$};
      \node[main, double, double distance=0.6mm] (eta) [right=of H] {$f_i$};
      \node[main, double, double distance=0.6mm] (ystar) [right=of eta] {$p_i$};
      \node[main, draw=colblu] (lambda) [above=of H, xshift=0.4cm, yshift=-0.4cm] {\color{colblu} $\lambda$};  %
      \node[main, draw=colgre] (alpha) [above=of eta, yshift=0.3cm] {\color{colgre} $\alpha$};  
      \node[main, fill = black!10] (y) [right=of ystar] {$y_i$};
      \node[main, draw=colred] (w) [below=of eta, yshift=0.4cm] {\color{colred} $w_i$};  
      \path (alpha) edge [connect] (eta)
            (lambda) edge [connect] (eta)
    		(H) edge [connect] (eta) 
    		(eta) edge [connect] node [above] {$\Phi$} (ystar)
    		(ystar) edge [connect] (y)
    		(w) edge [connect] (eta)
            (H) edge [] node [above] {$h$} (eta);
      \node[rectangle, draw=black!100, fit= (H) (y) (w) ] {}; 
      \node[rectangle, fit= (w) (y), label=below right:{$i=1,\dots,n$}, xshift=-0.35cm, yshift=0.55cm] {};  % the label
    \end{tikzpicture}
    
    \begin{textblock*}{0.48\textwidth}(0.52\textwidth,0.55cm)
    \begin{block}{}
    \vspace{-1.6em}
      \begin{align*}
        &p(\by,\bw,\alpha,\lambda)  \\
        &= p(\by|\bff)p(\bw)p(\lambda)p(\alpha) \\
        &= {\textstyle\prod_{i=1}^n} \Phi(f_i)^{y_i} \big(1 - \Phi(f_i)\big)^{1-y_i} \\
        &\phantom{==} \cdot {\color{colred} [\N(0,1)]^n} 
        \cdot {\color{colblu} \N(\lambda_0,\kappa_0^{-1})} \\
        &\phantom{==} \cdot {\color{colgre} \N(\alpha_0,\tau_0^{-1})}
      \end{align*}
    \end{block}
  \end{textblock*}
\end{frame}

\subsection{Laplace's method}
\begin{frame}{Laplace's method}
  \blfootnote{\fullcite[§4.1, pp. 777-778.]{Kass1995}}
  \vspace{-15pt}
  \begin{itemize}[<+->]\setlength\itemsep{0.8em}
    \item Interested in $p(\bff|\by) \propto p(\by|\bff)p(\bff) =: e^{Q(\bff)}$, with normalising constant $p(\by) = \int e^{Q(\bff)} \d\bff$. The Taylor expansion of $Q$ about its mode $\tilde\bff$
    \[
      Q(\bff) \approx Q(\tilde\bff) - \half (\bff - \tilde\bff)^\top\bA(\bff - \tilde\bff) 
    \]
    is recognised as the logarithm of an unnormalised Gaussian density, with $\bA = -\text{D}^2 Q(\bff)$ being the negative Hessian of $Q$ evaluated at  $\tilde\bff$.
    \item The posterior $p(\bff|\by)$ is approximated by $\N(\tilde\bff, \bA^{-1})$, and the marginal by
    \[
      p(\by) \approx (2\pi)^{n/2} \vert \bA \vert^{-1/2}  p(\by|\tilde\bff)p(\tilde\bff)
    \]
    \item Won't scale with large $n$; difficult to find modes in high dimensions.
  \end{itemize}
\end{frame}

\subsection{Full Bayesian analysis of I-probit models}
\begin{frame}{Full Bayesian analysis using MCMC}
  \vspace{-3pt}
  \begin{itemize}\setlength\itemsep{0.5em}
    \item Assign hyperpriors on parameters of the I-prior, e.g.
    \begin{itemize}
      \item $\lambda^2 \sim \Gamma^{-1}(a,b)$
      \item $\alpha \sim \N(c,d^2)$
    \end{itemize}
    for a hierarchical model to be estimated fully Bayes.
    \item No closed-form posteriors - need to resort to MCMC sampling.
    \item Computationally slow, and sampling difficulty results in unreliable posterior samples.
  \end{itemize}
  \begin{center}
    \includegraphics[scale=0.6]{mcmc1}
  \end{center}
\end{frame}