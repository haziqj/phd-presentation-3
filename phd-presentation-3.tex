\documentclass{beamer}
%\documentclass[xcolor={dvipsnames}, handout]{beamer} %for printing
\usepackage{haziq_beamer}
\bibliography{bib/phd-presentation-3}
\begin{document}

%%%%%%%%%%%%%%%%%%%%%%%%%%%%%%%%%%%%%%%%%%%%%%%%%%%%%%%%%%%%%%%%%%%%%%%%%%%%%%%%
%%% TITLE %%%%%%%%%%%%%%%%%%%%%%%%%%%%%%%%%%%%%%%%%%%%%%%%%%%%%%%%%%%%%%%%%%%%%%
%%%%%%%%%%%%%%%%%%%%%%%%%%%%%%%%%%%%%%%%%%%%%%%%%%%%%%%%%%%%%%%%%%%%%%%%%%%%%%%%

\title[I-prior probit]{Binary probit regression with I-priors}
\author[Haziq Jamil]{\large{Haziq Jamil}\\ \footnotesize{Supervisors: Dr. Wicher Bergsma \& Prof. Irini Moustaki}}
\institute[http://haziqj.ml]{Social Statistics (Year 3)\\ London School of Economics \& Political Science}
\date[8-9 May 2017]{8-9 May 2017\\
\hspace{1cm}\\
PhD Presentation Event\\
\hspace{1cm}\\
\href{http://phd3.haziqj.ml}{\color{fu-red} \textbf{http://phd3.haziqj.ml}}
}

\begin{frame}[plain]
  \addtocounter{framenumber}{-1}
  \titlepage
\end{frame}

%%%%%%%%%%%%%%%%%%%%%%%%%%%%%%%%%%%%%%%%%%%%%%%%%%%%%%%%%%%%%%%%%%%%%%%%%%%%%%%%
%%% TOC %%%%%%%%%%%%%%%%%%%%%%%%%%%%%%%%%%%%%%%%%%%%%%%%%%%%%%%%%%%%%%%%%%%%%%%%
%%%%%%%%%%%%%%%%%%%%%%%%%%%%%%%%%%%%%%%%%%%%%%%%%%%%%%%%%%%%%%%%%%%%%%%%%%%%%%%%

\mytoc

%%%%%%%%%%%%%%%%%%%%%%%%%%%%%%%%%%%%%%%%%%%%%%%%%%%%%%%%%%%%%%%%%%%%%%%%%%%%%%%%
%%% BODY %%%%%%%%%%%%%%%%%%%%%%%%%%%%%%%%%%%%%%%%%%%%%%%%%%%%%%%%%%%%%%%%%%%%%%%
%%%%%%%%%%%%%%%%%%%%%%%%%%%%%%%%%%%%%%%%%%%%%%%%%%%%%%%%%%%%%%%%%%%%%%%%%%%%%%%%

\section{Introduction}
%\subsection{I-priors}

\begin{frame}{The regression model}
  \vspace{-5pt}
  \begin{itemize}\setlength\itemsep{1em}
    \item For $i = 1, \dots, n$, consider the regression model
    \begin{align}
      \begin{gathered}
        y_i = f(x_i) + \epsilon_i \\
        (\epsilon_1, \dots, \epsilon_n) \sim \N(\bzero, \bPsi^{-1})
      \end{gathered}
    \end{align}
    where $f \in \cF$, $y_i \in \bbR$, and $x_i = (x_{i1}, \dots, x_{ip}) \in \cX$.
  \end{itemize}
  \begin{center}
    \includegraphics[scale=0.7]{figure/points}
  \end{center}
\end{frame}

\begin{frame}{I-priors}
  \vspace{-16pt}
  \blfootnote{\fullcite{Bergsma2017}}  
  \begin{itemize}\setlength\itemsep{0.3em}
    \item Let $\cF$ be a reproducing kernel Hilbert space (RKHS) with reproducing kernel $h_\lambda: \cX \times \cX \rightarrow \bbR$. An I-prior on $f$ is
    \[
      \big(f(x_1), \dots, f(x_n)\big)^\top \sim \N\big(\bff_0, \cI(f)\big),
    \] 
    with $\bff_0$ a prior mean, and $\cI$ the Fisher information for $f$, given by
    \[
      \cI\big(f(x), f(x')\big) = \sum_{k=1}^n\sum_{l=1}^n \psi_{kl} h_\lambda(x,x_k) h_\lambda(x',x_l).
    \]
    \pause
    \item The I-prior regression model for $i = 1,\dots,n$ becomes
    \begin{align}
      \begin{gathered}
        y_i = f_0(x_i) + \sum_{k=1}^n h_\lambda(x_i, x_k)w_k + \epsilon_i \\
        (w_1, \dots, w_n) \sim \N(\bzero, \bPsi) \\
        (\epsilon_1, \dots, \epsilon_n) \sim \N(\bzero, \bPsi^{-1}).
      \end{gathered}    
    \end{align}
  \end{itemize}
  \vspace{2pt}
\end{frame}

\begin{frame}{I-priors (cont.)}
  \blfootnote{\fullcite{Jamil2017}}  
  \vspace{-5pt}
  \begin{itemize}\setlength\itemsep{0.8em}
    \item Of interest is the posterior regression function characterised by the distribution
    \[
      p(\bff | \by ) = \frac{p(\by | \bff) p(\bff)}{\int p(\by | \bff) p(\bff) \d\bff} \pause,
    \]
    and also the posterior predictive distribution for new data points $x_{\text{new}}$
    \[
      p(y_{\text{new}} | \by) = \int p(y_{\text{new}} | \by, f_{\text{new}}) p(f_{\text{new}} | \by ) \d f_{\text{new}}
    \]
    with $f_{\text{new}} = f(x_{\text{new}})$.
    \pause
    \item Estimation using EM algorithm or direct maximisation of the marginal likelihood $\log p(y)$.
    \item Complete Bayesian estimation also possible.
  \end{itemize}
  \vspace{5pt}
\end{frame}

%\begin{frame}{Canonical/Linear RKHS}
%  \vspace{-5pt}
%  \only<1>{
%    \begin{center}
%      \includegraphics[scale=0.7]{figure/can-prior}
%    \end{center}
%  }
%  \only<2>{
%    \begin{center}
%      \includegraphics[scale=0.7]{figure/can-posterior}
%    \end{center}
%  }  
%\end{frame}

\begin{frame}{Fractional Brownian motion (FBM) RKHS}
  \vspace{-5pt}
  \only<1|handout:1>{
    \begin{center}
      \includegraphics[scale=0.7]{figure/fbm-prior}
    \end{center}
  }
  \only<2|handout:2>{
    \begin{center}
      \includegraphics[scale=0.7]{figure/fbm-posterior}
    \end{center}
  }  
  \only<3|handout:3>{
    \begin{center}
      \includegraphics[scale=0.7]{figure/fbm-posterior-truth}
    \end{center}
  } 
\end{frame}

\begin{frame}{Posterior predictive distribution}
  \vspace{-5pt}
  \only<1|handout:1>{
    \begin{center}
      \includegraphics[scale=0.7]{figure/credible-interval}
    \end{center}
  }
  \only<2|handout:2>{
    \begin{center}
      \includegraphics[scale=0.7]{figure/ppc}
    \end{center}
  }
\end{frame}

\subsection{PhD Roadmap}

\begin{frame}{PhD Roadmap}
  \vspace{-28pt}
  \only<1|handout:0>{
    \begin{center}
      \includegraphics[page=1,width=\textwidth,height=\textheight]{phd-roadmap}
    \end{center}
  }
  \only<2|handout:0>{
    \begin{center}
      \includegraphics[page=2,width=\textwidth,height=\textheight]{phd-roadmap}
    \end{center}
  }  
  \only<3|handout:0>{
    \begin{center}
      \includegraphics[page=3,width=\textwidth,height=\textheight]{phd-roadmap}
    \end{center}
  } 
  \only<4>{
    \begin{center}
      \includegraphics[page=4,width=\textwidth,height=\textheight]{phd-roadmap}
    \end{center}
  } 
\end{frame}







\section[Probit with I-priors]{Probit models with I-priors}
%\transition
\subsection{The latent variable motivation}

\begin{frame}{The latent variable motivation}
  \begin{itemize}
    \item Consider binary responses $y_1, \dots, y_n$ together with their corresponding covariates $x_1, \dots, x_n$. 
    \item For $i=1,\dots,n$, model the responses as
    \[
      y_i \sim \Bern(p_i).
    \]
    \item Assume that there exists continuous, underlying latent variables $y_1^*, \dots, y_n^*$, such that
    \[
      y_i =
      \begin{cases}
        1 & \text{ if } y_i^* \geq 0 \\
        0 & \text{ if } y_i^* < 0.    \\
      \end{cases}
    \]
    \item Model these continuous latent variables according to
    \[
      y_i^* = f(x_i) + \epsilon_i
    \]
    where $(\epsilon_1, \dots, \epsilon_n) \sim \N(\bzero, \bPsi^{-1})$ and $f \in \cF$ (some RKHS).
  \end{itemize}
\end{frame}

\subsection{Using I-priors}
\begin{frame}{Using I-priors}
  \begin{itemize}
    \item Assume an I-prior on $f$. Then,
    \begin{align*}
      \begin{gathered}
        f(x_i) = \alpha + \sum_{k=1}^n h_\lambda(x_i, x_k)w_k \\
        (w_1, \dots, w_n) \sim \N(\bzero, \bPsi) \\
      \end{gathered}
    \end{align*}
    \item For now, consider iid errors $\bPsi = \psi\bI_n$. In this case,
    \begin{align*}
      p_i = \Prob[y_i = 1] &= \Prob[y_i^* \geq 0] \\
      &= \Prob[\epsilon_i \leq f(x_i)] \\
      &= \Phi\Big(\psi^{1/2} ( 
%      {\color{gray} 
%      \underbrace{{\color{black} \alpha + {\textstyle\sum_{k=1}^n} h_\lambda(x_i, x_k)w_k}}_{\eta_i}
%      }
      \alpha + {\textstyle\sum_{k=1}^n} h_\lambda(x_i, x_k)w_k
      ) \Big)
    \end{align*}
    where $\Phi$ is the CDF of a standard normal.
    \item No loss of generality compared with using an arbitrary threshold $\tau$ or error precision $\psi$. Thus, set $\psi = 1$.
  \end{itemize}
\end{frame}

\begin{frame}{The probit I-prior model}
  \vspace{3pt}
  \begin{tikzpicture}[scale=1.1, transform shape]
    \tikzstyle{main}=[circle, minimum size = 10mm, thick, draw =black!80, node distance = 16mm]
    \tikzstyle{connect}=[-latex, thick]
    \tikzstyle{box}=[rectangle, draw=black!100]
      \node[main, draw=none] (fake) {};
      \node[main, fill = black!10] (H) [right=of fake, xshift=-1.65cm] {$x$};
      \node[main] (eta) [right=of H] {$f$};
      \node[main] (ystar) [right=of eta] {$y^*$};
      \node[main] (lambda) [above=of H, xshift=0.8cm, yshift=-0.5cm] {$\lambda$};
      \node[main] (alpha) [above=of eta, xshift=1.2cm, yshift=-0.5cm] {$\alpha$};  
      \node[main, fill = black!10] (y) [right=of ystar] {$y$};
      \node[main] (w) [below=of eta, yshift=0.5cm] {$w$};
    %  \node[main, fill = black!10] (x) [below=of eta,label=below:$x$] { };
      \path (alpha) edge [connect] (eta)
            (lambda) edge [connect] (eta)
    		(H) edge [connect] (eta) 
    		(eta) edge [connect] (ystar)
    		(ystar) edge [connect] (y)
    		(w) edge [connect] (eta);
      \path (H) edge [] node [above] {$h$} (eta);
      \node[rectangle, draw=black!100, fit= (H) (y) (w) ] {}; 
      \node[rectangle, fit= (w) (y), label=below right:N, xshift=1cm] {};  % the label
    \end{tikzpicture}
\end{frame}

\subsection{Estimation (and challenges)}

\begin{frame}{Estimation}
  \begin{columns}
    \uncover<1->{
    \begin{column}{0.47\textwidth}
      \vspace{6pt}
      \begin{itemize}\setlength\itemsep{0.5em}
        \item Denote $f_i = f(x_i)$ for short.
        \item The marginal density
        \vspace{4pt}
        \[
          \hspace{0.75cm} p(\by) = \int p(\by | \bff) p(\bff) \d\bff 
        \]
      \end{itemize}
    \end{column}}
    \uncover<4->{
    \begin{column}{0.5\textwidth}
    \vspace{-42pt}
      \begin{center}
        \includegraphics[scale=0.40]{figure/taylor_expand_meme}
      \end{center}
    \end{column}}
  \end{columns}
  \uncover<1->{
  \vspace{-5pt}
  \[
    \phantom{p(\by)} = \int \prod_{i=1}^n \left[ \Phi(f_i)^{y_i} \big(1 - \Phi(f_i)\big)^{1-y_i} \right] \cdot \N(\alpha\bone_n, \bH_\lambda^2) \d\bff
  \]
  \hspace{0.65cm} for which $p(\bff|\by)$ depends, cannot be evaluated analytically.}
  \vspace{3pt}
  \begin{itemize}
    \item<2-> Some strategies:
    \begin{itemize}
      \item[\xmark]<2-> Naive Monte-Carlo integral
      \item[\xmark]<3-> EM algorithm with a MCMC E-step
      \item[{\color{FUorange}\cmark}]<4-> Laplace approximation
      \item[{\color{FUorange}\cmark}]<5-> MCMC sampling
    \end{itemize}
  \end{itemize}
\end{frame}

%\begin{frame}
%  \begin{center}
%    \includegraphics[scale=0.5]{figure/taylor_expand_meme}
%  \end{center}
%\end{frame}

\subsection{What works}

\begin{frame}{Laplace's method}
  \blfootnote{\fullcite[§4.1, pp. 777-778.]{Kass1995}}
  \vspace{-15pt}
  \begin{itemize}\setlength\itemsep{0.8em}
    \item Interested in $p(\bff|\by) \propto p(\by|\bff)p(\bff) =: e^{Q(\bff)}$, with normalising constant $p(\by) = \int e^{Q(\bff)} \d\bff$. The Taylor expansion of $Q$ about its mode $\tilde\bff$
    \[
      Q(\bff) \approx Q(\tilde\bff) - \half (\bff - \tilde\bff)^\top\bA(\bff - \tilde\bff) 
    \]
    is recognised as the logarithm of an unnormalised Gaussian density, with $\bA = -\text{D}^2 Q(\bff)$ being the negative Hessian of $Q$ evaluated at  $\tilde\bff$.
    \item The posterior $p(\bff|\by)$ is approximated by $\N(\tilde\bff, \bA^{-1})$, and the marginal by
    \[
      p(\by) \approx (2\pi)^{n/2} \vert \bA \vert^{-1/2}  p(\by|\tilde\bff)p(\tilde\bff)
    \]
    \item Won't scale with large $n$; difficult to find modes in high dimensions.
  \end{itemize}
\end{frame}

\begin{frame}{Full Bayesian analysis using MCMC}
  \begin{itemize}\setlength\itemsep{0.5em}
    \item Assign hyperpriors on parameters of the I-prior, e.g.
    \begin{itemize}
      \item $\lambda^2 \sim \Gamma^{-1}(a,b)$
      \item $\alpha \sim \N(c,d^2)$
    \end{itemize}
    for a hierarchical model to be estimated fully Bayes.
    \item No closed-form posteriors - need to resort to MCMC sampling.
    \item Computationally slow, and sampling difficulty results in unreliable posterior samples.
  \end{itemize}
  *DENSITY PLOTS OF LAMBDA HERE*
\end{frame}










\section[Variational]{Variational inference}
%\transition
\subsection{Introduction}
\begin{frame}{Introduction}
  \blfootnote{\fullcite{Bishop2006}}
\end{frame}
\begin{frame}{Decomposition of the log marginal}
\end{frame}
\begin{frame}{Comparison}
\end{frame}
\begin{frame}{Factorised distributions (Mean field theory)}
\end{frame}
\begin{frame}{Variational Bayes EM}
\end{frame}

\subsection{A simple example}
\begin{frame}{Estimation of Gaussian mean and variance}
\end{frame}

\section{Illustration in R}
%\transition
%%\subsection{Toy example}
\begin{frame}{Simulated data}
\end{frame}
\begin{frame}{R code}
  Timings, parameter estimates, training error rate, test error rate
\end{frame}
\begin{frame}{Diagnostics}
  Monitor the lower bound
\end{frame}

\section{Applications}
%\transition
%%\begin{frame}{Cardiac arrhythmia data set}
\end{frame}
\begin{frame}{Multilevel example}
\end{frame}
\begin{frame}{Longitudinal example}
\end{frame}

\section{Summary}
%\transition
%%\begin{frame}{Summary}
\end{frame}
\begin{frame}{Way forward}
\end{frame}



\section*{End}

{
\framenonumber
\begin{frame}[noframenumbering]{End}
\begin{center}
\Huge Thank you!
\end{center}
\end{frame}
}

\appendix

%%%%%%%%%%%%%%%%%%%%%%%%%%%%%%%%%%%%%%%%%%%%%%%%%%%%%%%%%%%%%%%%%%%%%%%%%%%%%%%%
%%% REFERENCES %%%%%%%%%%%%%%%%%%%%%%%%%%%%%%%%%%%%%%%%%%%%%%%%%%%%%%%%%%%%%%%%%
%%%%%%%%%%%%%%%%%%%%%%%%%%%%%%%%%%%%%%%%%%%%%%%%%%%%%%%%%%%%%%%%%%%%%%%%%%%%%%%%

%\refslide

%%%%%%%%%%%%%%%%%%%%%%%%%%%%%%%%%%%%%%%%%%%%%%%%%%%%%%%%%%%%%%%%
%%%%%%%%%%%%%%%%%%%%%%%%%%%%%%%%%%%%%%%%%%%%%%%%%%%%%%%%%%%%%%%%
%\appendix
%\beginbackup
%%%%%%%%%%%%%%%%%%%%%%%%%%%%%%%%%%%%%%%%%%%%%%%%%%%%%%%%%%%%%%%%
%%%%%%%%%%%%%%%%%%%%%%%%%%%%%%%%%%%%%%%%%%%%%%%%%%%%%%%%%%%%%%%%
%
%{
%\framenonumber
%\begin{frame}[noframenumbering]
%	\tableofcontents[sections=6]
%\end{frame}
%}
%
%%%%%%%%%%%%%%%%%%%%%%%%%%%%%%%%%%%%%%%%%%%%%%%%%%%%%%%%%%%%%%%%%
%
%\section{Additional material}
%\subsection{g-priors}
%\begin{frame}{g-priors (1)}
%
%	\begin{itemize}\setlength\itemsep{1em}
%		\item g-priors (Zellner, 1986) for linear regression coefficients has covariance matrix proportional to the inverse Fisher information
%			\[
%				\boldsymbol\beta \sim \text{N}\big(\mathbf 0, g(\psi\mathbf X^\tpose\mathbf X)^{-1}\big)
%			\]
%		\item Popular choice of prior in Bayesian variable selection
%			\begin{itemize}
%				\item ``\textit{...use of $\propto (\mathbf X^\tpose\mathbf X)^{-1}$ tends to replicate design correlation}''
%		
%					(George and McCulloch, 1993)
%		
%				\item ``\textit{The choice of $\propto (\mathbf X^\tpose\mathbf X)^{-1}$ serves to replicate the covariance structure of the likelihood}'' (Chipman et. al., 2001)
%		
%				\item Used in applications such as gene selection (Lee et. al., 2003), disease staging (Sha et. al., 2004), crime data (Liang et. al., 2008), etc.
%			\end{itemize}
%	\end{itemize}
%
%\end{frame}
%
%%%%%%%%%%%%%%%%%%%%%%%%%%%%%%%%%%%%%%%%%%%%%%%%%%%%%%%%%%%%%%%%
%
%\begin{frame}{g-priors (2)}
%
%\begin{itemize}
%\item It is equivalent to using an independent prior on decorrelated data.
%\end{itemize}
%
%\begin{columns}
%    \begin{column}{0.33\textwidth}
%    	\begin{align*}
%		\left .
%        \begin{gathered} 
%        \mathbf y = \boldsymbol\alpha + \mathbf X \boldsymbol\beta + \boldsymbol\epsilon \\
%        \boldsymbol\epsilon \sim \text{N}(\mathbf 0, \psi^{-1}\mathbf I_n) \\
%        \boldsymbol\beta \sim \text{N}\big(\mathbf 0, g(\mathbf X^\tpose\mathbf X)^{-1}\big)
%        \end{gathered}
%        \color{white} \right \}
%        \hspace{-40pt}
%        \end{align*}
%	\end{column}
%    \begin{column}{0.05\textwidth}
%		$$\color{white} \Longleftrightarrow$$
%	\end{column}	
%    \begin{column}{0.33\textwidth}
%		\begin{align*} \color{white}
%		\hspace{-50pt}
%		\left \{
%        \begin{gathered} 
%        \mathbf y = \boldsymbol\alpha + \tilde{\mathbf X} \tilde{\boldsymbol\beta} + \boldsymbol\epsilon \\
%        \boldsymbol\epsilon \sim \text{N}(\mathbf 0, \psi^{-1}\mathbf I_n) 
%        \vspace{0.7em}\\                
%        \tilde{\mathbf X} = \mathbf X(\mathbf X^\tpose\mathbf X)^{-1/2} \\
%		\tilde{\boldsymbol\beta} = (\mathbf X^\tpose\mathbf X)^{1/2}\boldsymbol\beta \\
%        \tilde{\boldsymbol\beta} \sim \text{N}(\mathbf 0, g^2 \mathbf I_p)
%        \end{gathered}
%        \right .
%        \end{align*}    
%	\end{column}
%\end{columns}
%
%\begin{itemize}
%\item[ ] {\color{white} The intuition is the opposite of I-priors.
%
%\vspace{0.5em}
%\centering{
%%\hspace{-25pt}
%\textit{$\uparrow$ Fisher information $\Rightarrow$ $\downarrow$ variance $\Rightarrow$ prior is concentrated at zero}
%} }
%\end{itemize}
%
%\end{frame}
%
%%\miniframesoff
%\begin{frame}[noframenumbering]{g-priors (2)}
%
%\begin{itemize}
%\item It is equivalent to using an independent prior on decorrelated data.
%\end{itemize}
%
%\begin{columns}
%    \begin{column}{0.33\textwidth}
%    	\begin{align*}
%		\left .
%        \begin{gathered} 
%        \mathbf y = \boldsymbol\alpha + \mathbf X \boldsymbol\beta + \boldsymbol\epsilon \\
%        \boldsymbol\epsilon \sim \text{N}(\mathbf 0, \psi^{-1}\mathbf I_n) \\
%        \boldsymbol\beta \sim \text{N}\big(\mathbf 0, g(\mathbf X^\tpose\mathbf X)^{-1}\big)
%        \end{gathered}
%        \right \}
%        \hspace{-40pt}
%        \end{align*}
%	\end{column}
%    \begin{column}{0.05\textwidth}
%		$$\Longleftrightarrow$$
%	\end{column}	
%    \begin{column}{0.33\textwidth}
%    	\begin{align*}
%		\hspace{-50pt}
%		\left \{
%        \begin{gathered} 
%        \mathbf y = \boldsymbol\alpha + \tilde{\mathbf X} \tilde{\boldsymbol\beta} + \boldsymbol\epsilon \\
%        \boldsymbol\epsilon \sim \text{N}(\mathbf 0, \psi^{-1}\mathbf I_n) \\ 
%        \tilde{\boldsymbol\beta} \sim \text{N}(\mathbf 0, g^2 \mathbf I_p)  
%        \vspace{0.8em}\\                
%        \tilde{\mathbf X} = \mathbf X(\mathbf X^\tpose\mathbf X)^{-1/2} \\
%		\tilde{\boldsymbol\beta} = (\mathbf X^\tpose\mathbf X)^{1/2}\boldsymbol\beta \\
%        \end{gathered}
%        \right .
%        \end{align*}    
%	\end{column}
%\end{columns}
%\pause
%\begin{itemize}
%\item The intuition is the opposite of I-priors.
%
%\vspace{0.5em}
%\centering{
%%\hspace{-25pt}
%\textit{$\uparrow$ Fisher information $\Rightarrow$ $\downarrow$ variance $\Rightarrow$ prior is concentrated at zero}
%}
%\end{itemize}
%
%\end{frame}
%%\miniframeson
%
%
%\backupend

\end{document}